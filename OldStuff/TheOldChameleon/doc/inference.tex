

\documentclass{article}

\usepackage{latexsym}
\usepackage{amsmath}
\usepackage{code}

\pagestyle{plain}
\textwidth      150mm
\textheight     210mm
\oddsidemargin  -2mm
\evensidemargin -2mm

\newcommand{\LT}{LT}
\renewcommand{\implies}[0]{\supset}
%\newcommand{\implies}[0]{\rightarrow}
\newcommand{\arrow}[0]{\rightarrow}
\newcommand{\simparrow}[0]{\Longleftrightarrow} 
\newcommand{\proparrow}[0]{\Longrightarrow}
\newcommand{\rightarrowtail}{\longrightarrow}

\newcommand{\ms}[1]{%%\marginpar{\sc ms} 
                   {\bf MS:#1}}


\newtheorem{lemma}{Lemma}

\newtheorem{ex}{Example}
\newenvironment{example}{
        \begin{ex}\rm}%
        {\hfill$\Box$\end{ex}}

\newenvironment{ttline}{\begin{trivlist}\item \tt}{\end{trivlist}}
\newenvironment{ttprog}{\begin{trivlist} %%\small
            \item \tt
        \begin{tabbing}}{\end{tabbing}\end{trivlist}}

\newcommand{\gd}[0]{~\rule{0.5mm}{2.5mm}~}
\newcommand{\tcons}{\, \vdash_{\scriptsize C} \,}
\newcommand{\tdef}{\, \vdash_{\scriptsize R} \,}

\newcommand{\MRF}{\it MRF}
\newcommand{\ARF}{\it ARF}
\newcommand{\NRF}{\it NRF}


% Martin's macros
   

% groups

%%\newtheorem{lemma}{Lemma}
\newtheorem{definition}{Definition}
%%\newtheorem{theorem}{Theorem}
\newtheorem{corollary}{Corollary}
%%\newtheorem{remark}{Remark}
\newtheorem{conjecture}{Conjecture}
%%\newtheorem{example}{Example}

% arrays

\newcommand{\bi}{\begin{array}[t]{@{}l@{}}}
\newcommand{\ei}{\end{array}}
\newcommand{\ba}{\begin{array}}
\newcommand{\ea}{\end{array}}
\newcommand{\bda}{\[\ba}
\newcommand{\eda}{\ea\]}
\newcommand{\bp}{\begin{quote}\tt\begin{tabbing}}
\newcommand{\ep}{\end{tabbing}\end{quote}}
\newcommand{\set}[1]{\left\{
    \begin{array}{l}#1
    \end{array}
  \right\}}
\newcommand{\sset}[2]{\left\{~#1 \left|
      \begin{array}{l}#2\end{array}
    \right.     \right\}}


% symbols

\newcommand{\good}{good}
 
\newcommand{\err}{{\bf W}}
\newcommand{\VV}{{\cal V}} 
\newcommand{\GT}{{\cal T}_G}
\newcommand{\KK}{{\cal K}}
\newcommand{\TT}{{\cal T}}
\newcommand{\TC}{\mbox{\it TC}}
\newcommand{\UL}{\mbox{UL}}
\newcommand{\sUL}{\mbox{\scriptsize UL}}
\newcommand{\true}{\mbox{\sf true}}
\newcommand{\Int}{\mbox{\it Int}}
\newcommand{\Bool}{\mbox{\it Bool}}
\newcommand{\SF}{\mbox{$\cal S$}}
\newcommand{\tauvec}{\bar{\tau}}
\newcommand{\tvec}{\bar{t}}
\newcommand{\kvec}{\bar{k}}
\newcommand{\xvec}{\bar{x}}
\newcommand{\muvec}{\bar{\mu}}
\newcommand{\alphavec}{\bar{\alpha}}
\newcommand{\betavec}{\bar{\beta}}
\newcommand{\gammavec}{\bar{\gamma}}
\newcommand{\thetavec}{\bar{\theta}}
\newcommand{\deltavec}{\bar{\delta}}
\newcommand{\bvec}{\bar{b}}
\newcommand{\typo}{\Gamma}
\newcommand{\tenv}{\Gamma}
\newcommand{\typoinit}{\typo_0}
\newcommand{\static}{S}
\newcommand{\dynamic}{D}
\newcommand{\sstatic}{{\scriptstyle S}}
\newcommand{\sdynamic}{{\scriptstyle D}}
\newcommand{\BTA}{\mbox{BTA}}
\newcommand{\baseshape}{\bot}
\newcommand{\TCT}{P}
\newcommand{\Haskell}{H98}
\newcommand{\TCTinit}{P_o}

% figures, rules


\newcommand{\tlabel}[1]{\mbox{(#1)}}
\newcommand{\fig}[3]
        {\begin{figure*}[t]#3\
        \caption{\label{#1}#2}\ \hrulefill\ \end{figure*}}
\newcommand{\figurebox}[1]
        {\fbox{\begin{minipage}{\textwidth} #1 \end{minipage}}}
\newcommand{\boxfig}[3]
        {\begin{figure*}\figurebox{#3\caption{\label{#1}#2}}\end{figure*}}
\newcommand{\myirule}[2]{{\renewcommand{\arraystretch}{1.2}\ba{c} #1
                      \\ \hline #2 \ea}}

% relations, operators

\newcommand{\llist}[1]{\langle #1 \rangle}
\newcommand{\sat}{\mbox{\it sat}}
\newcommand{\entail}{\mbox{\it entail}}
\newcommand{\unambig}{\mbox{\it unambig}}
\newcommand{\complete}{\mbox{\it complete}}

\newcommand{\projexclusion}[1]{\mbox{$\bar{\exists}#1$}}
\newcommand{\funinst}{\mbox{\it inst}}
%%\renewcommand{\inst}{\, \vdash^{\scriptstyle i} \,}
%%\newcommand{\inst}{\, \vdash \,}
\newcommand{\ileq}{\preceq}
\newcommand{\ieq}{\simeq}
\newcommand{\topannot}[1]{|#1|}
\newcommand{\gendelta}{\mbox{\it gen}_{\delta}}
\newcommand{\genbeta}{\mbox{\it gen}_{\beta}}
\newcommand{\gen}{\mbox{\it gen}}
\newcommand{\normalize}{\mbox{\it normalize}}
\newcommand{\fail}{\mbox{\it fail}}
\newcommand{\fixpt}{\mbox{${\cal F}$}}
\newcommand{\testpa}{\mbox{${\cal T}$}}
\newcommand{\addpa}{\mbox{${\cal A}$}}
\newcommand{\matchpa}{\mbox{${\cal M}$}}
\newcommand{\refmatchpa}{\mbox{${\cal R}$}}
\newcommand{\wft}[1]{\mbox{\it wft}(#1)}
\newcommand{\tv}{\mbox{\it fv}}
\newcommand{\stv}{\mbox{\scriptsize fv}}
\newcommand{\range}{\mbox{\it range}}
\newcommand{\lub}{\wedge}
\newcommand{\tyrel}[2]{(#1 \preceq #2)}
\newcommand{\eqty}[2]{#1=#2}
%%\newcommand{\ts}{\, \vdash \,}
\newcommand{\turns}{\, \vdash \,}
\newcommand{\tsinst}{\, \vdash^i \,}
\newcommand{\tsttv}{\, \vdash^{ttv} \,}
\newcommand{\tsUL}{\, \vdash_{\mbox{\scriptsize UL}} \,}
\newcommand{\trc}{\, \vdash_{\scriptsize \id{inf}} \,}
\newcommand{\trl}{\, \vdash_{\scriptsize \id{fml}} \,}
\newcommand{\trcp}{\, \vdash_{\scriptstyle inf}' \,}
\newcommand{\genrel}{\, \vdash^{\scriptstyle gen} \,}
\newcommand{\asubty}[2]{(#1 \leq_a #2)}
\newcommand{\ssubty}[2]{(#1 \leq_s #2)}
\newcommand{\fsubty}[2]{(#1 \leq_f #2)}
\newcommand{\subty}[2]{(#1 \leq #2)}
\newcommand{\doubleb}[1]{[\![ #1 ]\!]}
\newcommand{\id}[1]{\mbox{\it #1}}


\newcommand{\mynote}[1]{$\spadesuit${\bf #1}$\clubsuit$}

% spacing

\newcommand{\sgap}{\quad}
\newcommand{\bgap}{\quad\quad}

% reserved words

\newcommand{\mathem}{\sf}
\newcommand{\IN}{\mbox{\mathem in}}
\newcommand{\LET}{\mbox{\mathem let}}
\newcommand{\MLET}{\mbox{\mathem mlet}}
\newcommand{\LETREC}{\mbox{\mathem letrec}}
\newcommand{\FIX}{\mbox{\mathem fix}}
\newcommand{\CASE}{\mbox{\mathem case}}
\newcommand{\TYCASE}{\mbox{\mathem typecase}}
\newcommand{\ELSE}{\mbox{\mathem else}}
\newcommand{\IF}{\mbox{\mathem if}}
\newcommand{\OF}{\mbox{\mathem of}}
\newcommand{\THEN}{\mbox{\mathem then}}
\newcommand{\INST}{\mbox{\mathem instance}}
\newcommand{\OVER}{\mbox{\mathem overload}}
\newcommand{\CLASS}{\mbox{\mathem class}}
\newcommand{\WHERE}{\mbox{\mathem where}}


%%% Local Variables: 
%%% mode: latex
%%% TeX-master: "bta"
%%% End: 


%%%%%%%%%%%%%%%%%%%%%%%%%%%%%%%%%%%%%%%%%%%%%%%%%%%%%%%%%%%%%%%%%%%%%%%%%%%%%%%%

\begin{document}

{\huge Type Inference via CHR Solving in Chameleon}

\tableofcontents

\section{Improved Inference for Checking Type Annotations}


\fig{f:constraint-lexical2}{Constraint and CHR Generation}{
\mbox{\bf Constraint Generation:} 
\bda{c}
 \tlabel{Var-x} ~~
  \myirule{ (x:t_1) \in \tenv \sgap \mbox{$t_2$ fresh}}
          {\tenv, E, x \tcons (t_2= t_1 \gd t_2)}
~~~~
 \tlabel{Var-f} ~~
  \myirule{\mbox{$t, l, l_g$ fresh} ~~\mbox{$(f \in E$ or} \\  \mbox{$f_a \in E$ $f$ non-recursive)} \\
           C = \{ f(t,l), l = (\llist{\overline{t_x}},l_g)\}}
   { \{ \overline{x:t_x} \} , E, 
  f \tcons ( C \gd t)}
\eda
\bda{c}
 \tlabel{VarA-f} ~~
  \myirule{\mbox{$f_a\in E$  $f$ recursive} \sgap \mbox{$t, l, l_g$ fresh} \sgap
           C = \{ f_a(t,l), l = (\llist{\overline{t_x},l_g})\}}
   {  \{ \overline{x:t_x} \} , E, 
  f \tcons ( C \gd t)}

\vspace{1mm}
\\ 
\tlabel{Abs} ~~
  \myirule{%%\tv(t_1) \subseteq V \\
            \tenv.x:t_1, E,e \tcons (C \gd t_2) \\ C'=\{C, t_3=t_1\arrow t_2 \} \sgap \mbox{$t_3$ fresh}}
          {\tenv, E,\lambda x::t_1.e \tcons (C' \gd t_3)}
~~~~
\tlabel{App} ~~ \myirule{\tenv, E,e_1 \tcons (C_1 \gd t_1) 
          \\ \tenv, E,e_2 \tcons (C_2 \gd t_2)
       \sgap \mbox{$t_3$ fresh}}
         {\tenv, E,e_1~e_2 \tcons 
       (C_1,C_2,t_1=t_2 \arrow t_3 \gd t_3)}

\vspace{1mm}
\\
\tlabel{Let} ~~ \myirule{\tenv, E\cup\{g\},e_2 \tcons (C \gd t)}
         {\tenv, E, \LET\ g = e_1 ~\IN\ e_2 \tcons (C \gd t)}
~~~~
\tlabel{LetA}  ~~
\myirule{        \tenv, E\cup \{g_a\}, e_2 \tcons (C \gd t)}
             {\tenv,   E,\LET\ 
             \ba{l} g :: C_1 \Rightarrow t_1 \\ g = e_1 \ea ~\IN\ e_2 
           \tcons (C \gd t)}
\eda
%
\mbox{\bf CHR Generation:} 
\bda{c}
  \tlabel{Var} ~~
  h, \tenv, E,v \tdef \emptyset
~~~~
 \tlabel{App} ~~
  \myirule{h, \tenv, E,e_1 \tdef P_1 \\ h, \tenv, E,e_2 \tdef P_2}
          {h, \tenv, E,e_1~e_2 \tdef P_1 \cup P_2}
~~~~
 \tlabel{Abs} ~~
 \myirule{h, \tenv.x:t, E,e \tdef P }
         {h, \tenv, E,\lambda x::t.e \tdef P}

\vspace{2mm}
\\ 
 \tlabel{Let} ~~
   \myirule{      \tenv = \{ x_1 : t_1,\ldots,x_n : t_n \}
           \sgap \mbox{$t,t_2',l,lr,lg$ fresh} \\ g,\tenv, E, e_1 \tdef P_1 
             \sgap h, \tenv, E\cup\{g\}, e_2 \tdef P_2 \sgap  \tenv, E,e_1 \tcons (C_1' \gd t_1') 
          \\ P = P_1 \cup P_2 \cup 
        \{ g(t,l) \simparrow C_1',t_1'=t,l=(\llist{t_1,\ldots,t_n | lr},\LT), h(t_2',l)_{\ominus} \} }
         {h, \tenv,E,\LET\ g = e_1 ~\IN\ e_2 \tdef P}

\vspace{2mm}
 \\ 
 \tlabel{LetA} ~~~
  \myirule{ 
            \tenv = \{ x_1 : t_1,\ldots,x_n : t_n \}            \sgap \mbox{$t,t_2',l,l_r$ fresh}
         \\ g, \tenv, E\cup\{g_a\}, e_1 \tdef P_1 \sgap h, \tenv, E\cup\{g_a\}, e_2 \tdef P_2
          \sgap \tenv, E\cup\{g_a\}, e_1 \tcons (C'_1 \gd t'_1)
          \\    P= P_1 \cup P_2 \cup \\
         \left \{ \ba{lcl}
       g_a(t,l) & \simparrow  & t=t''_1, C''_1, l=(\llist{t_1,...,t_n|lr},\LT), h(t_2',l)_{\ominus}
       \\ g(t,l) & \simparrow &   
       l=(\llist{t_1,\ldots.t_n | lr},\LT ), g_a(t,l), C'_1, t=t'_1 
           \ea
          \right \} 
       }
        {h, \tenv,   E,\LET\ 
             \ba{l} g :: C''_1 \Rightarrow t''_1 \\ g = e_1 \ea ~\IN\ e_2 \tdef P}
\eda
}

We give a brief description. Check ``Improved Inference for Checking Type Annotations'' (FP/chr/principal-annotations/tr.tex)
for details.
The translation to CHRs is described in Figure~\ref{f:constraint-lexical2}{Constraint and CHR Generation}.

\subsection{CHR Solving}

We assume simplified version of our justified CHR semantics.
We assume that each constraint is attached with either a $_{\ominus}$ marker 
or a $_{\epsilon}$ (pronounced empty) marker.
We refer to constraints carrying a $_{\ominus}$ marker as {\em marked} constraints.
Note that empty markers refer to justifications 
only containing positive numbers (we use numbers to represent justifications) whereas
marked constraints carry at least one negative number.


\begin{definition}[Marked CHR Application]
Let  $d= (U~\bar{t})_a$ (or $d= f(\bar{t})_a$) be a primitive constraint where $a\in \{ _{\ominus}, _{\epsilon} \}$. 
We define $mark(d)=a$. 
We write $d_{\ominus}$  to denote $(U~\bar{t})_{\ominus}$ (or $f(\bar{t})_{\ominus}$).

Let $(R)~~c_1,...,c_n \proparrow d_1, ..., d_m \in P$ and $C$ be a constraint.
                   Let $\phi$ be the m.g.u.~of all equations in $C$. Let $c'_1,...,c'_n\in C$ such that there exists a substitution $\theta$
                   on variables in rule (R) such that $\theta(c_i)=\phi(c'_i)$ for $i=1...n$. 
                    Then, $C \rightarrowtail_R C,d'_1,...,d'_m$
                   where \bda{lcl}  d'_i & = & 
                             \left \{ \ba{ll} d_i & \mbox{if $mark(c'_j) \not= {}_{\ominus}$ for $j=1...n$} \\
                                              (d_i)_{\ominus} & \mbox{otherwise}
                                      \ea
                             \right .
                         \eda

Let $(R)~~c \simparrow d_1, ..., d_m \in P$ and $C$ be a constraint.
                   Let $\phi$ be the m.g.u.~of all equations in $C$. Let $c'\in C$ such that there exists a substitution $\theta$
                   on variables in rule (R) such that $\theta(c)=\phi(c')$, that is user-defined constraint $c'$ {\em matches}
                   the left-hand side of rule (R). Then, $C \rightarrowtail_R C-c',d'_1,...,d'_m$
                   where $d'_i$ are as above.
\end{definition}


We denote by $\NRF$ the set of all non-recursive functions, by $\MRF$ the set of all recursive functions which carry no type annotations and
by $\ARF$ the set of all annotated recursive functions.

\begin{definition}[CHR Cycle Removal] 
Let $f(t,l)_{\ominus} \in C$ and $f(t',l')_{\ominus} \in C'$ where $f \in \NRF\cup\ARF$ and a derivation
            $... \rightarrowtail C \rightarrowtail ... \rightarrowtail C'$.
  Then, $... \rightarrowtail C \rightarrowtail ... \rightarrowtail C' \rightarrowtail_{CCR} C'-f(t',l')_{\ominus}$.


Let $f(t,l) \in C$ and $f(t',l')_{\ominus} \in C'$ where $f \in \MRF$ and a derivation
            $... \rightarrowtail C \rightarrowtail ... \rightarrowtail C'$.
  Then, $... \rightarrowtail C \rightarrowtail ... \rightarrowtail C' \rightarrowtail_{CCR} C'-f(t',l')_{\ominus},t'=t$.


Let $f(t,l) \in C$ and $f(t',l') \in C'$ where $f \in \MRF$ and $f(t',l')$ is
known to arise from the exact same program source location as $f(t,l)$, 
and given a derivation
            $... \rightarrowtail C \rightarrowtail ... \rightarrowtail C'$.
  Then, $... \rightarrowtail C \rightarrowtail ... \rightarrowtail C' \rightarrowtail_{Mono} C'-f(t',l'),t'=t$.

We assume that $\rightarrowtail_{CCR}$ and $\rightarrowtail_{Mono}$ are applied aggressively.
\end{definition}

\subsection{Type Inference via CHR Solving}

Consider type inference for an expression $e$ w.r.t.~an environment $\tenv$ of lambda-bound variables
and an environment $\tenv_{init}$ of primitive functions and
type class theory $P_p$. We assume $(P_{int},E_{int})$ model $\tenv_{init}$
such that for each $f:\forall \bar{a}.C \Rightarrow t'$ we find $f(t,l) \simparrow C,t=t' \in P_{init}$
and $f\in E_{init}$.
Then, we generate $\tenv, e \tcons (C \gd t)$ and 
$h,\tenv, E_{init}, e \tdef P_e$. We generally assume that $P$ denotes $P_p \cup P_e \cup P_{init}$.


For typability we need to check that (1) constraint $C$ is satisfiable, and
(2) all type annotations in $e$ are correct.
We are now in the position to describe CHR-based satisfiability and subsumption
check procedures.


\begin{definition}[Satisfiability Check] \label{def:sat} \mbox{} \\
Let $P$ be a set of CHRs and $C$ a constraint such that
$C \rightarrowtail^*_P C'$ for some constraint $C'$.
We say that $C$ is satisfiable iff the unifier of all equations in $C'$ exists.
\end{definition}

Let $E_{sub(e)}$ be the set of all predicate symbols $g_a$ where each $g_a$ refers to
some subexpression $(\LET\ g :: C_1 \Rightarrow t_1; g = e_1 ~\IN\ e_2)$ in $e$.
Let $F_{sub(e)}$ be a formula such that $\forall t,l.(g_a(t,l)\leftrightarrow g(t,l)) \in F_{sub(e)}$
for all $g_a \in E_{sub(e)}$.
It remains to verify that  the type annotation is correct under the abstraction of type inference in terms of $P$.
Formally, we need to verify that
$P \models F_{sub(e)}$ where $P$ refers to first-order logic 
interpretation of the set of CHRs $P$. 

We write $\bar{\exists}_{t,l}.C$ to denote
$\exists\tv(C)-\{t,l\}.C$.
%%MS: use now <-> from the start
%%Note that $(\forall t,l.g_a(t,l)\implies g(t,l))$ iff 
%%$(\forall t,l.g_a(t,l)\leftrightarrow g(t,l)$ w.r.t.~$P$. 
%%Recall that we include the annotation CHR on the rhs of each definition CHR.

\begin{definition}[Subsumption Check] \label{def:sub}
Let $g_a\in E_{sub(e)}$ and $P$ be a set of CHRs.
We say that {\tt g}'s annotation is correct iff
(1) we execute $g_a(t,l) \rightarrowtail^*_P C_1$
and $g(t,l) \rightarrowtail^*_P C_2$,
(2) we have that $\models (\bar{\exists}_{t,l}.C_1) \leftrightarrow  (\bar{\exists}_{t,l}.C_2)$.
\end{definition}

\begin{definition}[Building of Type Schemes]
Let $P$ be a set of CHRs, {\tt g} a function symbol.
We say function {\tt g} has type $\forall\alphavec.C'\Rightarrow t'$ w.r.t.~$P$ iff
(1) We have that $g(t, l) \rightarrowtail^* C,l=(l_l,l_g)$ for some constraint $C$,
(2) $\phi$, the m.g.u.~of $C,l=(l_l,l_g)$ exists,  
(3) let $D\subseteq \phi(C)$ such that
$D$ is maximal and $D$ consists of unmarked user-defined constraints only,
(4) let $\bar{a} = \tv(D, \phi t) - \tv(\phi l_l)$,
(5) let $C'=\phi(C)$ and $t'=\phi(t)$.
\end{definition}

\subsection{Variations}

\subsubsection{``Strict'' Inference}
\label{sec:strict}

We may not constrain the types of lambda-bound variables in case we never ``call'' a function.

\begin{example}
Consider:

\begin{ttprog}
f x = \= let \=g = x x \\
      \> in  \>x
\end{ttprog}

We generate CHR rules that look like the following:
\bda{rcl}
f(t) & \simparrow & t = t_x \arrow t_x \\
g(t) & \simparrow & t_x = t_x \arrow t
\eda

Since the $f$ rule never calls $g$, the unsatisfiable constraint is not
introduced, and this program is considered well-typed.

\end{example}

This ``laziness'' can be problematic whenever we need to compare the inferred 
type of some function with its declared type.
Consider an annotated function $g$, nested within the definition of function
$f$, from which we generate an inference rule, 
$g(t,l) \simparrow g_a(t,l), C_i$, and an annotation rule, 
$g_a(t,l) \simparrow f(t',l')_{\ominus}, C_a$. 
It's possible that the types of some global variables are affected in $g$, but
not in $g_a$. 
In order for $g(t,l)$ and $g_a(t,l)$ to be equivalent, we depend on $g$'s
context, as called in the $g_a$ rule, to in turn call $g$ and introduce those
missing constraints.

\begin{example}
\label{ex:ann-ctxt}
Consider the following program.

\begin{ttprog}
f y = \=let \=g~::~Bool \\
      \>    \>g = y     \\
      \>in  \>'a'
\end{ttprog}

We generate the following (simplified) CHR rules:
\bda{rcl}
f(t,l) & \simparrow & t = Char,~l = (\llist{}, \llist{t_y}) \\
g(t,l) & \simparrow & t = t_y,~l = (\llist{t_y}, \llist{t_y}),~g_a(t,l) \\
g_a(t,l) & \simparrow & t = Bool,~l = (\llist{t_y}, \llist{t_y}),~f(t',l) 
\eda

Even though {\tt g}'s type annotation is acceptable, out subsumption check
would fail, because $g(t,l)$ and $g_a(t,l)$ are not equivalent wrt the $l$
component. 
Clearly, the $g$ rule implies $t_y = Bool$, but the $g_a$ rule does not.

\end{example}

We can remedy this situation by ensuring that all nested functions are called 
by their parent function. In this way, when we consider a function annotation,
the definition of the function becomes part of its context.

\begin{example}
We modify the program of Example~\ref{ex:ann-ctxt}, forcing {\tt f} to call
{\tt g}, but disregard its value.

\begin{ttprog}
f y = \=let \=g~::~Bool \\
      \>    \>g = y     \\
      \>in  \>const 'a' g
\end{ttprog}

The CHR rule associated with {\tt f} would now look something like:
\bda{rcl}
f(t,l) & \simparrow & t_{const} = a \arrow b \arrow a, a = Char, t = t_y \arrow a, \\
       &            & l = (\llist{}, \llist{t_y}),~g(b,l')
\eda

This solves our immediate problem, in that the constraints arising from
$g(t,l)$ and $g_a(t,l)$ are now equialent wrt $t$ and $l$.

Unfortunately this does not work in the case where the function we call is
recursive. Consider:

\begin{ttprog}
f y = \=let \=g~::~Bool \\
      \>    \>g = const y g \\
      \>in  \>'a' 
\end{ttprog}

Here, if {\tt f} were to call {\tt g}, the constraint generated to represent
{\tt g}'s type would be $g_a(t',l')$.
We then face the same problem, that from within $g_a$ we have no association
between $t_y$ and $Bool$.

\end{example}

Clearly, a syntactic transformation of the source program to introduce calls
to otherwise uncalled functions is not sufficient.
We must modify the CHR generation process to directly insert calls to the
inference constraints of functions which are not already called.

\begin{example}
We return to the rules generated in Example~\ref{ex:ann-ctxt}.
The following CHR rule is a modified version of the $f$ rule above which now
contains a call to $g$, further constraining the lambda-bound type variables.

\bda{rcl}
f(t,l) & \simparrow & t = Char,~l = (\llist{}, \llist{t_y}),~g(t',l)_{\ominus}
\eda

Using this modified rule, we see now that $g(t,l)$ and $g_a(t,l)$, are
equivalent, since $t_y = Bool$ in both.
The $\ominus$ mark on the $g$ constraint is not significant here, though it
does accomodates the simpler form of cycle breaking (by simply removing the
repeated constraint) than the equivalent unmarked constraint would.
\end{example}

\subsubsection{Alternative Translation Scheme: Add $f_a$ to rhs of definition CHR}


(Material can be found in FP/chr/principal-annotations/main.tex)
Such a translation scheme is strictly weaker.
Imagine examples
where the inner annotation $f_a$ needs some global information provided by $p$.

\begin{example} \label{ex:spurious}
Consider
\begin{ttprog}
class Erk a where erk ::~a \\
class Bar a where bar ::~a  \\
f = (erk, \= let \= g ::~Foo a=>a \\
       \>\>       g = bar \\
     \>     in g)
\end{ttprog}
\end{example}
%
In some artificial type class extension we impose the condition
that $P_p = \{ (R)~~Erk ~a, Foo ~b \proparrow Bar~b \}$.
Here is a sketch of the translation to CHRs.
\bda{rcl}
 g_a(t) & \simparrow & f(t')_{\ominus}, Foo~t \\
 g(t)   & \simparrow & g_a(t), Bar~t \\
 f(t)   & \simparrow & t=(a,b), Erk~a, g(b)
\eda
%
We find that
\bda{ll}
 & g(t_1) \\
\rightarrowtail_g & g_a(t_1), Bar~t_1 \\
\rightarrowtail_{g_a} & f(t')_{\ominus}, Foo~t_1, Bar~t_1 \\
\rightarrowtail_{f}      & \mbox{(add ${}_\ominus$ to rhs)} \\
 & t'=(a,b), (Erk~a)_{\ominus}, g(b)_{\ominus}, \\ & Foo~t_1, Bar~t_1 \\ 
\rightarrowtail_g & t'=(a,b), (Erk~a)_{\ominus}, g_a(b)_{\ominus}, (Bar~b)_{\ominus}, \\ & Foo~t_1, Bar~t_1 \\ 
\rightarrowtail_{g_a} & t'=(a,b), (Erk~a)_{\ominus}, f(t'')_{\ominus}, \\ & (Foo~b)_{\ominus}, (Bar~b)_{\ominus}, Foo~t_1, Bar~t_1 \\ 
\rightarrowtail_{CCR} & t'=(a,b), (Erk~a)_{\ominus}, (Foo~b)_{\ominus}, \\ & (Bar~b)_{\ominus}, Foo~t_1, Bar~t_1 \\ 
\rightarrowtail_R & t'=(a,b), (Erk~a)_{\ominus}, (Foo~b)_{\ominus}, \\ & (Bar~b)_{\ominus}, Foo~t_1, Bar~t_1, (Bar~t_1)_{\ominus}
\eda
%
\ms{We unnecessarily break the cycle too late, no harm, but blows up example a bit.}
Furthermore, we have that
\bda{ll}
 & g_a(t_1) \\
\rightarrowtail_{g_a} & f(s')_{\ominus}, Foo~t_1 \\
\rightarrowtail_{f} & s'=(c,d), (Erk~c)_{\ominus}, g(d)_{\ominus}, Foo~t_1 \\
\rightarrowtail_{g} &  s'=(c,d), (Erk~c)_{\ominus}, g_a(d)_{\ominus}, \\ & (Bar~d)_{\ominus}, Foo~t_1 \\
\rightarrowtail_{g_a} & s'=(c,d), (Erk~c)_{\ominus}, f(s'')_{\ominus}, (Foo~d)_{\ominus}, \\ & (Bar~d)_{\ominus}, Foo~t_1 \\
\rightarrowtail_{CCR} & s'=(c,d), (Erk~c)_{\ominus}, (Foo~d)_{\ominus}, \\ & (Bar~d)_{\ominus}, Foo~t_1 \\
\rightarrowtail_R & s'=(c,d), (Erk~c)_{\ominus}, (Foo~d)_{\ominus}, \\ & (Bar~d)_{\ominus}, Foo~t_1, (Bar~t_1)_{\ominus} 
\eda
%
Note that final stores of both derivations are logically equivalent modulo variable $t_1$.
E.g. we can rename $s'$ to $t'$ and so on. Hence, {\tt g}'s type annotation is correct.
More importantly, it is crucial to include $f(t')_{\ominus}$ on the rhs of 
the annotation CHR as suggested by Transformation 3.
Performing a translation to CHRs guided by Transformation 3' would yield
$g_a(t_1) \rightarrowtail^* Foo~t_1$. Hence, the subsumption check fails.


\subsubsection{Alternative Translation Scheme: Stricly use $f_a$ (if $f$ non-recursive) in case of $f$}

Again weaker scheme, see Ex20 ( ref (ex:ex)) in FP/chr/principal-annotatins/main.tex.


\section{Lexically Scoped Type Annotations}

See FP/chr/principal-annotations/main.tex for details.

\fig{f:lexical-constraint}{Constraint Generation for Lexically Scoped Annotations}{

\bda{c}
\ba{cc}

% \tlabel{VarA-f} &
%  \myirule{\mbox{$f_a \in E$  $f$ recursive $t, l, l_g, v, v_l, vr$ fresh} \\
%           C = \{f_a(t,l,v), l = (\llist{\overline{t_x}},l_g), 
%                      v=(v_l,\llist{\bar{a} |vr})\}}
%   { \bar{a}, \{\overline{x:t_x} \}, E,
%  f \tcons ( C \gd t)}

 \tlabel{VarA-f} &

  \myirule{\mbox{$f_a^{\bar{c}} \in E$  $f$ recursive $t, l, l_g, v, v_l$ fresh} \\

           C = \{f_a(t,l,v), l = (\llist{\overline{t_x}},l_g), 
                      v=(v_l,\llist{\bar{c}})\}}
   { \bar{a}, \{\overline{x:t_x} \}, E,
  f \tcons ( C \gd t)}



\ea
\\ \\
\ba{cc}
 \tlabel{Var-x} & 
  \myirule{ (x:t_1) \in \tenv \sgap \mbox{$t_2$ fresh}}
          {V, \tenv, E, x \tcons (t_2= t_1 \gd t_2)}
\ea
 ~~~~~
\ba{cc}
 \tlabel{Var-f} &
  \myirule{\mbox{$(f^{\bar{c}}\in E)$ or ($f_a^{\bar{c}} \in E$ $f$ non-recursive)} \\ \mbox{$t, l, l_g, v, v_l$ fresh} \\
           C = \{ f(t,l,v), l = (\llist{\overline{t_x}},l_g), 
                      v=(v_l,\llist{\bar{c}})\}}
   { \bar{a}, \{\overline{x:t_x} \}, E,
  f \tcons ( C \gd t)}
\ea
\\ \\
\ba{cc}
\tlabel{Abs} &
  \myirule{%%\tv(t_1) \subseteq V \\
            V, \tenv.x:t_1, E, e \tcons (C \gd t_2) \\ \mbox{$t_3$ fresh}}
          {V, \tenv, E, \lambda x::t_1.e \tcons (C, t_3=t_1\arrow t_2 \gd t_3)}
\ea
  ~~~~~
\ba{cc}
\tlabel{App} & \myirule{V, \tenv, E, e_1 \tcons (C_1 \gd t_1) 
          \\ V,\tenv, E, e_2 \tcons (C_2 \gd t_2)
       \sgap \mbox{$t_3$ fresh}}
         {V,\tenv, E, e_1~e_2 \tcons 
       (C_1,C_2,t_1=t_2 \arrow t_3 \gd t_3)}
\ea
\\ \\
\ba{cc}
\tlabel{Let} & \myirule{V,\tenv, E \cup \{ g \}, e_2 \tcons (C \gd t)}
         {V,\tenv, E, \LET\ g = e_1 ~\IN\ e_2 \tcons (C \gd t)}
\ea
  ~~~~~
\ba{cc}
\tlabel{LetA}  &
\myirule{        V, \tenv, E \cup \{ g_a \}, e_2 \tcons (C \gd t)}
             {V, \tenv,   E, \LET\ 
             \ba{l} g :: C_1 \Rightarrow t_1 \\ g = e_1 \ea ~\IN\ e_2 
           \tcons (C \gd t)}
\ea
\\ \\
\ba{cc}
\tlabel{LetA2} &
\myirule{V, \Gamma, E, \LET \ba{l} g :: C_1 \Rightarrow t_1 \\ g = e_1 \ea
                      ~\IN\ e_2 \tcons (C \gd t)}
        {V, \Gamma, E, \LET \ba{l} g ::: C_1 \Rightarrow t_1 \\ g = e_1 \ea
                      ~\IN\ e_2 \tcons (C \gd t)}
\ea

\eda
}


\fig{f:lexical-rule}{Rule Generation for Lexically Scoped Type Annotations}{
\bda{c}
\ba{cc}
  \tlabel{Var} &
  h^{\bar{c}}, V, \tenv, E, v \tdef \emptyset
\ea
~~~~
\ba{cc}
 \tlabel{App} & 
  \myirule{h^{\bar{c}},V, \tenv, E, e_1 \tdef P_1 \\ h^{\bar{c}}, V, \tenv, E, e_2 \tdef P_2}
          {h^{\bar{c}},V, \tenv, E, e_1~e_2 \tdef P_1 \cup P_2}
\ea
~~~~
\ba{cc}
 \tlabel{Abs} &
 \myirule{h^{\bar{c}},V, \tenv.x:t, E, e \tdef P }
         {h^{\bar{c}},V, \tenv, E, \lambda x::t.e \tdef P}
\ea
\\ \\
\ba{cc}
 \tlabel{Let} & 
   \myirule{ \tenv = \{ x_1 : t_1,\ldots,x_n : t_n \}
           \sgap V = \{b_1,\ldots,b_m \}  \sgap \mbox{$t,t_2',l,v,v',vl,vl', lr$ fresh} 
         \\ g^{\bar{b}},V, \tenv, E, e_1 \tdef P_1 \sgap h^{\bar{c}},V, \tenv, E \cup \{g^{\bar{b}}\}, e_2 \tdef P_2 \sgap  
                 V, \tenv, E, e_1 \tcons (C \gd t') 
                   \\ \ba{llcl} P = P_1 \cup P_2 \cup & 
                \{ g(t,l,v) & \simparrow & C,t'=t,
                 l=(\llist{t_1,\ldots.t_n | lr},\LT), v=(vl,\llist{b_1,...,b_m}),\\
              & & & h(t_2',l,v')_{\ominus}, v'=(vl',\llist{\bar{c}}) \} 
          \ea}
         {h^{\bar{c}},V, \tenv, E, \LET\ g = e_1 ~\IN\ e_2 \tdef P}
\ea
 \\ \\
\ba{cc}
 \tlabel{LetA} ~~~~~~~\mbox{}&
  \myirule{ \{a_1,\ldots,a_k\} = \tv(C''_1,t''_1) - \tv(V,\tenv) 
             \sgap \tenv = \{ x_1 : t_1,\ldots,x_n : t_n \} \sgap
            V = \{b_1, \ldots, b_m \}
           \\ \mbox{$lr, t,t_2',l,l_2,v,v',vl',v_2$ fresh} \sgap vl'=V-\bar{c}
         \\ g^{\bar{b}}, V.\bar{a}, \tenv, E \cup \{g_a^{\bar{b}}\}, e_1 \tdef P_1 \sgap 
               h^{\bar{c}},V, \tenv, E \cup \{g_a^{\bar{b}}\}, e_2 \tdef P_2
          \sgap V.\bar{a}, \tenv, E \cup \{g_a^{\bar{b}}\}, e_1 \tcons (C'_1 \gd t'_1)

          \\    P= P_1 \cup P_2 \cup \\
         \left \{ \ba{lcl}
       g_a(t,l,v) & \simparrow  & t=t''_1, C''_1, l=(\llist{t_1,...,t_n|lr},\LT), \\
            &&
           v= (\llist{a_1,\ldots,a_k},\llist{b_1,\ldots,b_m}),
          h(t_2',l,v')_{\ominus},v'=(vl',\llist{\bar{c}})
       \\ g(t,l,v) & \simparrow &   
       l=(\llist{t_1,\ldots.t_n | lr},\LT ), v= (\llist{a_1,\ldots,a_k},\llist{b_1,\ldots,b_m}), 
  \\ && g_a(t,l,v), C'_1, t=t'_1 
           \ea
          \right \} 
       }
        {h^{\bar{c}},V, \tenv,   E, \LET\ 
             \ba{l} g :: C''_1 \Rightarrow t''_1 \\ g = e_1 \ea ~\IN\ e_2 \tdef P}
\ea \\

\ba{cc}
 \tlabel{LetA2} ~~~~~~~\mbox{}&
  \myirule{ \{a_1,\ldots,a_k\} = \tv(C''_1,t''_1) - \tv(V,\tenv) 
             \sgap \tenv = \{ x_1 : t_1,\ldots,x_n : t_n \} \sgap
            V = \{b_1, \ldots, b_m \}
           \\ \mbox{$lr, t,t_2',l,l_2,v,v',vl',v_2$ fresh} \sgap vl'=V-\bar{c}
         \\ g^{\bar{b}}, V.\bar{a}, \tenv, E \cup \{g^{\bar{b}}\}, e_1 \tdef P_1 \sgap 
           h^{\bar{c}},V, \tenv, E \cup \{g^{\bar{b}}\}, e_2 \tdef P_2
          \sgap V.\bar{a}, \tenv, E \cup \{g^{\bar{b}}\}, e_1 \tcons (C'_1 \gd t'_1)

          \\    P= P_1 \cup P_2 \cup \\
         \left \{ \ba{lcl}
       g(t,l,v) & \simparrow &   
       l=(\llist{t_1,\ldots.t_n | lr},\LT ), 
       v= (\llist{a_1,\ldots,a_k},\llist{b_1,\ldots,b_m}), 
  \\ && t=t''_1, C''_1, C'_1, t=t'_1, h(t_2',l,v')_{\ominus},v'=(vl',\llist{\bar{c}})
           \ea
          \right \} 
       }
        {h^{\bar{c}},V, \tenv,   E, \LET\ 
             \ba{l} g ::: C''_1 \Rightarrow t''_1 \\ g = e_1 \ea ~\IN\ e_2 \tdef P}
\ea \\

\eda
}

With the addition of scoped annotations, the subsumption check described in
Definition~\ref{def:sub} must be modified slightly. 
The new check is as follows.

We generate the set of CHRs $P$ from a program, and check the inferred
type of function $g$ against its declared type by executing 
$g_a(t,l,v) \rightarrowtail^*_P C_1$ and 
$g(t,l,v)   \rightarrowtail^*_P C_2$, where 
$C_1 = C_1' \cup \{l = (l_{l_1}, l_{g_1})\}$ and
$C_2 = C_2' \cup \{l = (l_{l_2}, l_{g_2})\}$. 
Let $C_1'' = C_1' \cup \{l = (l_{l_1}, a_1)\}$ and
    $C_2'' = C_2' \cup \{l = (l_{l_2}, a_2)\}$, where $a_1$ and $a_2$ are fresh
variables.
We then check that
$\models (\bar{\exists}_{t,l,v}. C_1'') \leftrightarrow 
         (\bar{\exists}_{t,l,v}. C_2'')$.

Essentially we are replacing the $LT$ component of the resulting constraints by 
fresh variables before testing equivalence.
The reason we need to do this is that there may be constraints in $C_2$, but
not in $C_1$, which affect variables bound within $g$, and therefore 
appear {\em only} within the $LT$ component.
We only want to test the equivalence of $C_1$ and $C_2$ wrt. unbound
variables.



\begin{example}
Consider the following.

\begin{ttprog}
test y = \= let \= f~::~d -> Bool \\
         \>     \> f$_1$ x = isEmpty (y >> return x) \\
         \> in  f y
\end{ttprog}

\newcommand{\ellist}{\langle \rangle}

We generate the following (simplified) CHRs from this program.
\bda{rcl}
test(t,l,v) & \simparrow & t=t_y=t_r,l=(\_,LT), v=(\_,\llist{}), \\
            &            & f(t_y\arrow t_r,(\llist{t_y},LT),(\_,\llist{})), \\ 
f_a(t,l,v)  & \simparrow & t=d \arrow Bool, l=(\llist{t_y|\_},LT), \\ 
            &            & v=(\llist{d},\llist{}), test(\_,l,v')_{\ominus},v'=(\_,\llist{}) \\
f(t,l,v)    & \simparrow & f_a(t,l,v), t=t_x \arrow a, \\
            &            & l=(\llist{ty|\_},LT), v=(\llist{d},\llist{}), ...
\eda

We run $f_a(t,l,v) \rightarrowtail^* C_1$ and $f(t,l,v) \rightarrowtail^* C_2$.
Projecting $C_1$ onto $t,l,v$, we get the following
$t = d \arrow Bool, l = (\llist{t_y|\_}, \llist{t_y,t_x}), v=(\llist{d},\llist{}),...$. 
Similarly, for $C_2$ we get
$t = d' \arrow Bool, l = (\llist{t_y'|\_}, \llist{t_y',\underline{d'}}), v=(\llist{d'},\llist{}), ...$
Note that constraints are not logically equivalent (modulo variables $t$, $l$ and $v$) due to
the underlined constraint. The definition CHR establishes the fact that $d'$ and $t_x'$ are equal.
This is expected, since by simple inspection of the above program we see that
$d$ and the type of {\tt x} are the same.
However, the annotation CHR cannot establish this connection.
Clearly, lambda-bound variables which are locally defined should not matter when testing for subsumption.
Hence, we replace the $LT$ component by some fresh variables.
Effectively, we compare 
$t = d \arrow Bool, l = (\llist{t_y|\_}, \_), v=(\llist{d},\llist{}),...$
an $t = d' \arrow Bool, l = (\llist{t_y'|\_}, \_), v=(\llist{d'},\llist{}), ...$
which are logically equivalent.
\end{example}


We note that all results stated carry over to lexically scoped type annotations.
An interesting observation is that our trick to infer as much context information when
checking type annotations is in particular valuable for lexically scoped type annotations.

\ms{We also say more about the subtle interaction between {\tt ::} and {\tt :::}.}


\section{Type Inference via Implication Constraints}



\fig{f:constraint}{Constraint Generation with Implications}{
\bda{c}
 \tlabel{Var-x} ~~
  \myirule{ (x:t_1) \in \tenv \sgap \mbox{$t_2$ fresh}}
          {\tenv, E, x \tcons (t_2= t_1 \gd t_2)}
~~~~
 \tlabel{Var-f} ~~
  \myirule{\mbox{$t, l, l_g$ fresh} ~~f \in E \\  
           C = \{ f(t,l), l = \llist{\overline{t_x}}\}}
   { \{ \overline{x:t_x} \} , E, 
  f \tcons ( C \gd t)}
~~~~
 \tlabel{VarA-f} ~~
  \myirule{\mbox{$t, l, l_g$ fresh} ~~f_a\in E  \\ 
           C = \{ f_a(t,l), l = \llist{\overline{t_x}}\}}
   {  \{ \overline{x:t_x} \} , E, 
  f \tcons ( C \gd t)}

\vspace{1mm}
\\ 
\tlabel{Abs} ~~
  \myirule{%%\tv(t_1) \subseteq V \\
            \tenv.x:t_1, E,e \tcons (C \gd t_2) \\ \mbox{$t_3$ fresh}}
          {\tenv, E,\lambda x::t_1.e \tcons (C, t_3=t_1\arrow t_2 \gd t_3)}
~~~~
\tlabel{App} ~~ \myirule{\tenv, E,e_1 \tcons (C_1 \gd t_1) 
          \\ \tenv, E,e_2 \tcons (C_2 \gd t_2)
       \sgap \mbox{$t_3$ fresh}}
         {\tenv, E,e_1~e_2 \tcons 
       (C_1,C_2,t_1=t_2 \arrow t_3 \gd t_3)}

\vspace{1mm}
\\
\tlabel{Let} ~~ \myirule{\tenv, E\cup\{g\},e \tcons (C \gd t) \\ \tenv=\overline{x:t} \sgap \mbox{$t'$ fresh}}
         {\tenv, E, \LET\ [g~p_i = e_i]_{i\in I} ~\IN\ e \tcons (g(t',\llist{\bar{t}})_{\ominus},C \gd t)}
~~~~
\tlabel{LetA}  ~~
\myirule{        \tenv, E\cup \{g_a\}, e \tcons (C \gd t) \\
                 \mbox{$t'$, $l$, $l_g$ fresh} \\
                  \tenv=\overline{x:t} \sgap \bar{a}=\tv(C_1,t_1)  \\
                  C' = \{ g(t',\llist{\bar{t}})_{\ominus}, C, ((C_1,t=t_1,l=\llist{\bar{t}}) \implies^{\bar{a}} g(t,l)) \} }
             {\tenv,   E,\LET\ 
             \ba{l} g :: C_1 \Rightarrow t_1 \\ \mbox{$[$}g~p_i = e_i\mbox{$]$}_{i \in I}  \ea ~\IN\ e 
           \tcons (C' \gd t)}
\vspace{1mm}
\\
\tlabel{Case} ~~
 \myirule{ \tenv, E, p_i \arrow e_i \tcons (C_i \gd t_i) \sgap \mbox{for $i\in I$}  \\
         \tenv, E, e \tcons (C_e \gd t_e) \sgap \mbox{$t_1, t_2$ fresh} \\ 
         C = \{ C_e, t_1=t_e \arrow t_2 \} \cup
           \bigcup_{i\in I} \{ C_i, t_1=t_i \} }
         {\tenv, E, \CASE\ e~ \OF\ [p_i \arrow  e_i]_{i\in I} \tcons (C \gd t_2)}
~~~~
\tlabel{Pat} ~~
\myirule{p \tcons \forall \bar{b} (D \gd \tenv_p), (C_p \gd t_p) \\ 
          \tenv \cup \tenv_p, e \tcons (C_e \gd t_e)   \\
          \tenv = \overline{x:t} \sgap \mbox{$t$ fresh} \\
            C  =  \{ (D \implies^{\bar{b}} C_e), 
                t=t_p\arrow t_e, C_p \} }
         {\tenv, E, p \arrow e \tcons (C \gd t)}
\eda

\bda{cc}
 \ba{cc}
   \tlabel{Pat-Var} &
   \myirule{\mbox{$t$ fresh}}
         {x \tcons (True \gd \{x:t\}), (True \gd t)}
\ea
 &
\ba{cc}
 \tlabel{Pat-Pair} &
  \myirule{p_1 \tcons \forall \overline{b_1}(D_1 \gd \tenv_{p_1}), (C_1 \gd t_1) \\
           p_2  \tcons \forall \overline{b_2}(D_2 \gd \tenv_{p_2}), (C_2 \gd t_2)}
          {(p_1,p_2)  \tcons
      \forall \overline{b_1,b_2} (D_1,D_2 \gd \tenv_{p_1}\cup \tenv_{p_1}), (C_1,C_2 \gd (t_1,t_2))}
\ea
\eda
\bda{cc}
 \tlabel{Pat-K} &
 \myirule{K:\forall\bar{a},\bar{b}.D \Rightarrow t \arrow T~\bar{a} \sgap
          \\ %%\bar{b} \cap \bar{a} = \emptyset \\          
          p \tcons \forall \bar{b'}(D' \gd \tenv_p), (C_p \gd t_p)}
         {K~p \tcons \forall \bar{b'},\bar{b} (D',D \gd \tenv_p), (C_p,t=t_p \gd T~\bar{a})}
\eda

}

\fig{f:rule}{CHR Generation with Implications}{
\bda{c}
  \tlabel{Var} ~~
  \tenv, E,v \tdef \emptyset
~~~~
 \tlabel{App} ~~
  \myirule{\tenv, E,e_1 \tdef P_1 \\ \tenv, E,e_2 \tdef P_2}
          {\tenv, E,e_1~e_2 \tdef P_1 \cup P_2}
~~~~
 \tlabel{Abs} ~~
 \myirule{\tenv.x:t, E,e \tdef P }
         {\tenv, E,\lambda x::t.e \tdef P}

\vspace{2mm}
\\ 
\tlabel{Case} ~~
 \myirule{ \tenv, E, e_i \tdef P_i \sgap \mbox{for $i\in I$}  \\
           \tenv, E, e \tdef P_e \\
          P = \bigcup_{i\in I} P_i \cup P_e}
         {\tenv, E, \CASE\ e~ \OF\ [p_i \arrow  e_i]_{i\in I} \tdef P}

\vspace{2mm}
\\ 
 \tlabel{Let} ~~
\myirule{\tenv = \overline{x:t} \sgap \mbox{$t,t_2',l,lr$ fresh} \\
         \tenv, E, e_i \tdef P_i \sgap 
         \tenv, E, p_i \arrow e_i \tcons (C_i \gd t_i) \sgap i\in I \\
         \tenv, E \cup \{ g \}, e' \tdef P' \\ 
       P = \bigcup_{i\in I} P_i  \cup P' \cup \\ 
        \{ g(t,l) \simparrow (\bigcup_{i\in I} C_i,t_i=t) 
                 ,l=\llist{t_1,\ldots,t_n | lr} \} }
        {\tenv, E, \LET\ [g~p_i = e_i]_{i\in I} ~\IN\ e' \tdef P}

\vspace{2mm}
 \\ 
 \tlabel{LetA} ~~~
  \myirule{ 
              \tenv = \overline{x:t}         \sgap \mbox{$t,t_2',l,l_r$ fresh} \\
      \tenv, E\cup\{g_a\}, e_i \tdef P_i \sgap 
         \tenv, E\cup\{g_a\}, p_i \arrow e_i \tcons (C_i \gd t_i) \sgap i\in I \\
            \tenv, E\cup\{g_a\}, e \tdef P'
          \sgap \tenv, E\cup\{g_a\}, e_1 \tcons (C'_1 \gd t'_1)
          \\    P= \bigcup_{i\in I} P_i  \cup P' \cup \\
         \left \{ \ba{lcl}
       g_a(t,l) & \simparrow  & t=t''_1, C''_1 \\ %%, l=(\llist{t_1,...,t_n|lr},\LT) \\
       g(t,l) & \simparrow & (\bigcup_{i\in I} C_i,,t_i=t) 
                 ,l=\llist{t_1,\ldots,t_n | lr}
           \ea
          \right \} 
       }
        {\tenv,   E,\LET\ 
             \ba{l} g :: C''_1 \Rightarrow t''_1 \\  \mbox{$[$}g~p_i = e_i\mbox{$]$}_{i \in I} 
                      \ea ~\IN\ e \tdef P}
\eda
}


See Figures~\ref{f:constraint} and \ref{f:rule}.
Note that we check for subsumption locally by generating implication constraints.
Hence, there's no need anymore to include the ``context'' $h(t_2',l)_{\ominus}$.
Hence, there's no need anymore for the ``global'' $l$ component.
Note that the constraint generation rules already enforce ``strict'' inference.
We call $g(t',\llist{\bar{t}})_{\ominus}$ in rules \tlabel{Let} and \tlabel{LetA}.
That's the only reason now why we need ``marked'' constraints.
Optimization: In case there's already a call to $g$, we can omit $g(t',\llist{\bar{t}})_{\ominus}$.
Note that in case there's a call to $g_a$ omitting $g(t',\llist{\bar{t}})_{\ominus}$ may yield
a weaker system.
Consider
\begin{code}
rule Bar b, Foo a ==> Bar a
f = (erk, let g :: Foo a => a
              g = bar
          in g)
\end{code}
%
In $f$'s CHR we find the constraint
$Erk~c, \forall a. (Foo~a \implies Bar~a)$.
If we use $g$ instead of $g_a$ we also find $Foo~b,Bar~b$. Hence, we can solve the implication constraint.

Performing type inference: Given $(\tenv,e)$ build $\tenv, E e \tdef P$ and $\tenv, E, e \tcons (C \gd t)$.
Solve $C$ w.r.t. $P$. In case $e$ consists of definitions of the form {\tt f = e} and
{\tt f::C=>t;f=e} we simply consider {\tt let $e$ in undefined}.


Performing type inference for inner definitions:
Assume (in the interactive environment) we ask for {\tt :t f1;...;fn;g}.
We solve $g(t,l), f1(t1,l)_{\ominus},...,fn(tn,l)_{\ominus}$.

\section{Coinduction}

We apply the following meta-rule.
Assume $ ... \rightarrowtail ... C \rightarrowtail ... \rightarrowtail C'$
such that $a_{i,J}\in C$ and $a_{j,J'}\in C'$ such that $a$ is a user-defined constraint,
$J$ is a suffix of $J'$ ($J$ may be the emptyset) and $i$ and $j$ are distinct.
Then, $ ... \rightarrowtail ... C \rightarrowtail ... \rightarrowtail C' \rightarrowtail_{CoInd} C'-a_{j,J'}, True_{j,J'-J,i}$.

The above rule preserves the CHR soundness result.
Note that $a_{j,J'}$ must be (logically) implied by some constraints in the store.

Rule falls out naturally if we turn all simplification rules into propagation rules and enforce set semantics.

Some ``strange'' cases. Consider $(Eq~a)_1, (Eq~a)_2 \rightarrowtail^* (Eq~a)_1, True_{2,1}$.
Fine. Could happen in case of ``redundant'' type annotations.




\end{document}
